\section*{執筆にあたって}

これは第16回KEKサマーチャレンジ(ksc16)のレポートに紙面の都合上掲載が叶わなかった内容を補完するためのものである。
公式から刊行されるレポートは僅か2ページと制限が強い。
短期間であるとはいえ、本演習で学んだ内容を余すことなく記すには、公式のレポートとは別の枠が必要である。
ここでは紙面の制限をかけることなく、自由闊達に記述していきたい。

当然、分量に糸目をつけないからには完成までに長い時間が必要である。
ともすれば公式の報告が発刊されてなお、この文書の完成を見ることは難しいであろう。
そこで製作途中であっても随時公開していくこととする。
最終版の完成まで、読者には適宜アップデートをしてもらう形になるが、上記の都合を鑑みてご容赦願いたい。

また筆者の経験からして、過去の演習の記録を参照できないのは非常に歯痒いものがあった。
これがksc17以降に役立てば幸いである。


高エネルギー加速器研究機構 (KEK) で開催される素粒子・原子核のサマーチャレンジも今年で16回を数える。
かつての参加者が教官として学生を教えるようになった。


決して詰ろうというのではない。
参加者はおよそ50人にもなる。
2ページのレポートでさえ纏めてしまえば読むに堪えない長大なものとなる。
研究を細部まで掴めるほど細密な記述をしてしまうと、組み合わせればレポートに慣れた事務方も目を当てられない代物が出来上がるだろう。


かつては発表会のあとにポスターセッションが設けられ、25分の発表時間に収まりきらない議論を交わす場となっていたという。

帰宅してレポートのため冷えた頭で再び解析すると思わぬ発見や考察の再考などに見舞われる。
解析や考察は必然膨大になり、わずか2ページに収めることはできない。

これが独善的な理由である。

ここでのレポートは趣旨が異なる。

通常の大学や専門学校であれば半期をかけるような演習を、わずか1週間で仕上げなければならない。
夜になれば運営から停止の指令が電話でかかる。
圧倒的に時間が足りない。

どの班にも発展的な課題が用意されているが、

過去のレポートがあるのなら何故見せてくれないものか。

このレポートが ksc17 以下、演習の参考になってくれれば、執筆者冥利に尽きる。
