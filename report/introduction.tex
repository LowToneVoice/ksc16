\section*{執筆にあたって}

高エネルギー加速器研究機構 (KEK) で開催される素粒子・原子核のサマーチャレンジも今年で16回を数える。
かつての参加者が教官として学生を教えるようになった。


決して詰ろうというのではない。
参加者はおよそ50人にもなる。
2ページのレポートでさえ纏めてしまえば読むに堪えない長大なものとなる。
研究を細部まで掴めるほど細密な記述をしてしまうと、組み合わせればレポートに慣れた事務方も目を当てられない代物が出来上がるだろう。

ここでのレポートは趣旨が異なる。

通常の大学や専門学校であれば半期をかけるような演習を、わずか1週間で仕上げなければならない。
夜になれば運営から停止の指令が電話でかかる。
圧倒的に時間が足りない。

どの班にも発展的な課題が用意されているが、

このレポートが ksc17 以下、演習の参考になってくれれば、執筆者冥利に尽きる。
