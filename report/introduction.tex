\section*{編集にあたって}

高エネルギー加速器研究機構(KEK)で開催される素粒子・原子核のサマーチャレンジも今年で16回を数える.
北海道から沖縄まで津々浦々の大学生・高専生が筑波に集まり1週間余をかけて演習を行うこの企画も、かつての参加者が教官として学生を教えるようになった.
その演習の濃密さたるや、無理が祟って体重が減る参加者が毎年のように現れるという.

本文書は第16回KEKサマーチャレンジ(ksc16)の公式レポートに紙面の都合上掲載が叶わなかった内容を補完するためのものである.
公式に刊行されるレポートは1人あたり僅か2ページと制限が強く、短期間であるとはいえ本演習で学んだ内容を余すことなく記すには公式レポートとは別の枠が必要である.
ここでは紙面の制限をかけることなく、自由闊達に記述していきたい.
当然、分量に糸目をつけないからには完成までに長い時間が必要である.
ともすれば公式の報告が発刊されてなお、この文書の完成を見ることは難しいであろう.
そこで製作途中であってもGitHub上で随時公開していくこととする.
最終版の完成まで読者には適宜アップデートをしてもらう形になるが、上記の都合を鑑みてご容赦願いたい.

編者の体験からして、過去の演習記録を参照できないのは非常に歯痒いものがあった.
学生実験として実験デザインや解析手法、その他問題解決はできる限り学生の手で行うという趣旨に共感はできるが、実際の研究が過去の成果をもとにアップデートしていくものである以上、またわずか1週間では到底用意されている演習をクリアできない以上、我々の演習成果を翌年以降につなげることには労力以上の意義があると考えている.
本文書がksc17以降の演習に役立てば本望である.
KEKサマーチャレンジ関係者以外でも大いに参考となるであろう.
ここで取り扱っている実験はどれも素粒子・原子核の基礎的な実験でありながら、学生にとって学ぶところの多いようにデザインされている.
比類なく濃密な実験ができる場としてサマーチャレンジ参加の判断材料にしてもらっても良い.
これを機にサマーチャレンジに応募する読者が現れれば、この上ない喜びである.

% 最後に、編者の我儘な呼びかけに何人もの参加者に協力していただいた.
% 誰かに頼まれたわけでもなく、中には\LaTeX を使った文書作成も初めてだった執筆者も多くいた中で、これほどクオリティの高いものが出来上がったのには驚かされるばかりである.
% 末筆ながらここで御礼申し上げる.

\rightline{\today}

\rightline{編者}
